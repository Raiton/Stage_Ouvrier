\chapter{Présentation de l'entreprise}
\section{Historique}
Carthage Power Company (CPC) a été fondée en 1996 sous la forme d'une Société Anonyme à Responsabilité Limitée.
 Elle avait la tâche de construire une centrale électrique à cycle combiné de 471 MW de puissance suite à l'appel d'offre lancé par le ministère de l'industrie.
 
La centrale, située à la zone pétrolière  à environ 15 Km de Tunis, a été construite par ALSTOM suite à un investissement d'environ 261 Millions de Dollars pour une cession  de 20 ans à compter de la date de mise en service.

Le capital de la société est  réparti sur deux associés :

\begin{itemize}

\item 60 \% détenue par  BTU
\item le reste (40 \%) détenue par Marubeni (Une maison de commerce japonaise)

\end{itemize}

Carthage Power Company est la seule société privée produisant de l'électricité en Tunisie.

Elle assure environ  23 \% de la production nationale fournie par la Société Tunisienne de l'électricité et du Gaz STEG.

Du fait du contrat de cession d'électricité, la CPC vend exclusivement  l'énergie à la STEG. Cette dernière constitue en même temps son unique fournisseur en gaz naturel, principal combustible de la production.

\section{Organisation hiérarchique}

La CPC compte plusieurs services qui accomplissent des tâches différentes et complémentaires. On trouve dans un premier niveau les départements administratifs et support à la société : direction technique, ressources humaines, finance...

Dans un deuxième niveau, on retrouve le corps principal de l'activité de la société qui gère la production et la maintenance de la centrale électrique. Cette dernière est dirigée par un Plant Manager  et compte plusieurs départements : Département Mécanique, Département EI\&C ({Electrique, Instrumentation et Contrôle commande), Département Technique (études et planification), Département Exploitation... \\
Pour plus de détails sur  sur l'organisation de la société, consulter l'Annexe \uppercase\expandafter{\romannumeral 1} 

\begin{comment}
La CPC emploie au total 70 personnes, dont un chef de centrale, deux responsables d'exploitation et de maintenance, six administrateurs.\\Le reste du staff est constitué par des ingénieurs et des techniciens de formation réparti comme suit :
\end{comment}

La société fait aussi appel aux services des contractants permanents et non permanants (SOMMI, General Electric, ALSTOM) pour assurer des taches spécifiques de maintenance des équipement de la centrale.

Le staff de l'exploitation et de la maintenance est constitué par des ingénieurs et des techniciens de formation réparti comme suit :

\begin{itemize}
\item Mécanique
\item Électromécanique
\item Électrique 
\item Automatique et régulation 
\item Instrumentation et maintenance industrielle
\item Chimie
\item Informatique
\end{itemize}


Le personnel d'exploitation est réparti en six équipes pour assurer l'exploitation de la centrale en 24h/24h. Ce même personnel exécute les travaux de maintenance durant une période dans leurs cycles de rotation en plus du personnel permanent de maintenance pour diminuer les effets de la routine et acquérir plus de compétences. 

	\begin{comment}
	
	La CPC est organisée selon une hiérarchie horizontale où tout le monde (ou presque) s'implique lors de la réalisation d'une tâche. En effet, la maintenance est répartie sur toutes les équipes de travail, et tous les responsables s'en occupent ce qui atténue énormément la charge individuelle. 
	\end{comment}
\section{Certifications}
      CPC a entamé, depuis le mois de novembre 2005, la préparation du système de management intégré qui inclut la qualité, la sécurité  et l'environnement. 
      
        En 2006 CPC a obtenue la certification en Management intégré de qualité sécurité et environnement  ISO 9001, OHSAS 18001, ISO 14001.
        
        De telles accomplissements montrent bien l'engagement de la CPC et sa direction générale pour garantir la sécurité des employés et la zone,  la protection de l'environnement et la qualité de la production.   