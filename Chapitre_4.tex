\chapter{Aspect social}
Durant ma période de stage, j'ai eu l'occasion d'être en contact continu avec les différentes unités de travail. J'ai pu acquérir des connaissances sur l'environnement professionnel et les relations interpersonnels au sein de l'entreprise
\section{Environnement de travail}
\subsection{Horaires}
La journée de travail à la centrale  commence à $7^h$ et finit à $15^h:30$ avec une pause déjeuner d'une demi-heure. Au mois de Ramadan, l'arrêt du travail est en général à 15 h.

La société accorde une grande importance à la ponctualité des employés. En effet, ces derniers effectuent un pointage électronique à chaque entrée ou sortie de la centrale. Toute absence est enregistrée. Le système de pointage est utilisé aussi pour des raisons de sécurité pour pouvoir identifier touts les personnes présent dans la centrale si une évacuation est nécessaire en cas d'accident dans la centrale.
\subsection{Sécurité}
Les risques dans la centrale thermique sont toujours présents et de natures diverses.
Il est donc primordial de prendre des mesures de sécurité pour la protection au sein du site.

En effet, des essais périodiques sont effectués sur  des organes, équipements et systèmes de détection et de lutte contre incendie  pour assurer leurs disponibilités.

Des consignes de sécurité strictes sont à appliquer avant et pendant l'exécution des divers travaux de maintenance. En effet tout travail de maintenance ou d'exploitation nécessite la signature d'un SAFE WORK PERMIT par un supérieur  (voir Annexe \uppercase\expandafter{\romannumeral 7}) pour assurer un travail en toute sécurité. Ce permis décrit les conditions des travaux ainsi que les points à consigner.

En outre, le port des équipements de sécurité est obligatoire dans les endroits indiqués (Voir Annexe \uppercase\expandafter{\romannumeral 6}).

Le personnel est tenu de participer  à des formations en matière de sécurité telle que l'école du feu de la STIR et secourisme (Protection Civile). Moi même j'ai regardé une vidéo décrivant les dangers potentiels au sein du site et les précautions à prendre.

%L'entreprise est dotée d'un plan de gestion de crise (Plan d'Opération Interne (P.O.I.)) pour définir toutes les actions à prendre dans le cas d'un grave accident. 

\section{Relations interpersonnels}
\subsection{Relations horizontales}
Les relations entre techniciens sont détendues en générale. Un esprit de solidarité et coopération règne dans toute la centrale. 
 
Les employés se partagent les expériences et les idées dans une ambiance de travail sérieuse. N'empêche que l'atmosphère est parfois détendue par des plaisanteries qui aide les employés   à échapper à la monotonie et stress du travail.

Cet environnement conviviale m'a facilité l'intégration au sein de cette équipe. 

\subsection{Relations verticales}
Les relations entre techniciens et leurs supérieurs hiérarchiques sont régis essentiellement  par  le respect et la bonne entente. L'aspect autoritaire n'est pas ressenti.

 Les employés communiquent de façon directe et ouverte avec leurs supérieurs hiérarchiques. Ces derniers peuvent leurs donner conseils et les aider dans les tâches compliquées. 
 
Cette bonne entente contribue à la diminution du stress et l'amélioration de l'efficacité et le rendement de travail.  
 
 